%
\documentclass[11pt]{report}

\usepackage{graphicx}
\usepackage{braket}
\usepackage{mdframed}
\usepackage{bm}
\usepackage{amssymb}
\usepackage{amsthm}
\usepackage{amsmath}
\usepackage{enumerate}
\usepackage{amsmath}
\usepackage{mathrsfs}
\usepackage{multicol}
\usepackage{verbatim}
\usepackage{dsfont}
\usepackage{color}

%\usepackage[T1]{fontenc}
%\usepackage[euler-digits,small]{eulervm}
%\usepackage[sc,osf]{mathpazo}
%\linespread{1.025}
%\usepackage[utf8]{inputenc}

\usepackage{geometry}
\geometry{
letterpaper,
lmargin=2cm,
rmargin=2cm,
tmargin=2cm,
bmargin=2cm,
footskip=12pt,
headheight=12pt}

\newcommand{\iden}{\mathds{1}}

\newcommand{\Z}{\mathbb Z}
\newcommand{\T}{\mathbb{T}}
\newcommand{\lcm}{\text{lcm}}
\newcommand{\NN}{\mathbb{N}}
\newcommand{\QQ}{\mathbb{Q}}
\newcommand{\RR}{\mathbb{R}}
\newcommand{\CC}{\mathbb{C}}
\newcommand{\KK}{\mathbb{K}}
\newcommand{\DD}{\mathbb{D}}
\newcommand{\VV}{\mc {V}}
\newcommand{\WW}{\mc {W}}
\newcommand{\FF}{{\mathbb{F}}}
\newcommand{\tat}{\text}
\newcommand{\ti}{\to\infty}
\newcommand{\mc}{\mathcal}
\newcommand{\ms}{\mathscr}
\newcommand{\mf}{\mathfrak}
\newcommand{\mbb}{\mathbb}
\newcommand{\cnj}{\overline}
\newcommand{\veps}{\varepsilon}
\newcommand{\sg}{\sigma}

\newcommand{\restr}{\upharpoonright}
\newcommand{\FR}[2]{\frac{#1}{#2}}
\newcommand{\PP}[2]{\frac{\di #1}{\di #2}}
\newcommand{\PFR}[2]{\left(\frac{#1}{#2}\right)}
\newcommand{\SFR}[2]{\sqrt{\frac{#1}{#2}}}

\newcommand{\ten}{\otimes}

\newcommand{\Align}[1]{\begin{align*}#1\end{align*}}

\theoremstyle{plain}
\newtheorem{thm}{Theorem}[section] 
\newtheorem{lem}[thm]{Lemma}

\theoremstyle{definition}
\newtheorem{defn}{Definition}
\newtheorem{proposition}{Proposition}[section]
\newtheorem{conj}{Conjecture}
\newtheorem{ex}{Example}

\theoremstyle{remark}
\newtheorem{remark}{Remark}[section]
\newtheorem*{note}{Note}
\newtheorem{case}{Case}
\newtheorem*{claim}{Claim}

\DeclareMathOperator{\spec}{Spec}
\DeclareMathOperator{\Card}{card}
\DeclareMathOperator{\Span}{span}
\DeclareMathOperator{\rank}{rank}
\DeclareMathOperator{\real}{Re}
\DeclareMathOperator{\diam}{diam}
\DeclareMathOperator{\id}{id}
\DeclareMathOperator{\GL}{GL}

\def\sm{\setminus}
\def\seq{\subseteq}
\def\ii{\item}
\def\bE{\begin{enumerate}}
\def\eE{\end{enumerate}}
\def\bP{\begin{pmatrix}}
\def\eP{\end{pmatrix}}

\renewcommand{\empty}{\varnothing}
\newcommand{\hv}[1]{\hat{\textbf{#1}}}
\newcommand{\oti}[2]{#1_{#2=1}^\infty}
\newcommand{\lam}{\lambda}
\newcommand{\om}{\omega}
\newcommand{\Om}{\Omega}
\newcommand{\gam}{\gamma}
\newcommand{\di}{\partial}

\newcommand{\ha}{\hat a}
\newcommand{\had}{\hat a^\dagger}

\newcommand{\colr}[1]{ {\color{red}  #1 } }
\newcommand{\colb}[1]{ {\color{blue} #1 } }
\newcommand{\colnb}[1]{ {\color{NavyBlue} #1 } }
\newcommand{\colm}[1]{ {\color{Fuchsia} #1 } }

\date{\today}
\title{Gravitation and Cosmology Notes}
\author{}

\begin{document}
\maketitle

\chapter{Special Relativity}

The axioms of Special Relativity are:
\begin{enumerate}
\item The laws are invariant under change of inertial reference frames.
\item In inertial reference frames, there is an absolute speed of signal
propogation, $c=1$.
\end{enumerate}

\section{Lorentz Transformations}
Let $x^\mu,x'^\mu$ be two coordinate systems and $c=1$. For now we work in the
classical vacuum and here experiments\footnote{Is there a better explanation?}
point us to the fact that light travels at the absolute speed $c$.
Suppose $A,B$ are spacetime events representing emmission and absorption of
light. By the second postulate, the distance between these two points is the
same in both reference frames, and in particular, the quantity 
\[(\Delta x^0)^2-(\Delta x^1)^2-(\Delta x^2)^2-(\Delta x^3)^2 = 0 = (\Delta
x'^0)^2-(\Delta x'^1)^2-(\Delta x'^2)^2-(\Delta x'^3)^2.\]
This leads us to the definition of the \textbf{proper time}:
\[d\tau^2 := dt^2 - d\bm x^2 = -\eta_{\alpha\beta}dx^\alpha dx^\beta.\]

The notation $dx^\alpha dx^\beta$, in the mathematical sense, is a simple tensor
$dx^\alpha\ten dx^\beta$ and \emph{not} the symmetric tensor. Using the Einstein
summation convention and the fact that our metrics are always \emph{symmetric},
this means that the contraction $\eta_{\alpha\beta}dx^\alpha dx^\beta
=\eta_{00}dx^0\ten dx^0+\eta_{10}dx^1\ten dx^0+\eta_{01}dx^0\ten{dx}^1+\cdots
=\eta_{00}dx^0\ten dx^0+\eta_{10}(dx^1\ten dx^0+dx^0\ten{dx}^1)+\cdots$. Now we
return back to proper time and see how far we can run with the concept.

\begin{proposition} Let $\ms L \seq \GL_\RR(3,1)$ denote the subgroup of linear
operators that fix proper time: $d\tau'^2 = d\tau^2$. Then the set $\ms L$ can
be described concretely:
\[ \ms L = \{ \Lambda\ |\ \Lambda^\alpha_\gamma
\Lambda^\beta_\delta\eta_{\alpha\beta} = \eta_{\beta\delta}\}.\]
\end{proposition}
\begin{proof}
The proof of this is a straightforward and we follow Weinberg's derivation.
Take a linear transformation $x'^\alpha = \Lambda^\alpha_\beta x^\beta$.
Then $d\tau^2 = d\tau'^2$ equivalent to:
\begin{align*}
    -\eta_{\beta\delta} dx^\beta dx^\delta
  &=-\eta_{\alpha\gamma} dx'^\alpha dx'^\gamma\\
  &=-\eta_{\alpha\gamma}(\Lambda^\alpha_\beta dx^\beta)
                        (\Lambda^\gamma_\delta dx^\delta)
\end{align*}
Therefore, $\eta_{\beta\delta} = \eta_{\alpha\gamma} \Lambda^\alpha_\beta
\Lambda^\gamma_\delta$ as was desired. 
\end{proof}

\begin{remark}
An important remark is in order. The above proposition emphasized
\emph{linear} transformations. In fact it is possible to show that the set
of all nonsingular analytic coordinate transformations (ones which are
invertible and locally given by a convergent power series) coincides with
affine transformations of the form $x' = \Lambda x + a$, with
$\Lambda\in\ms L$.  To show this it suffices to show the second derivative
of the coordinate transformation is identically zero, which implies that
all higher order derivatives also vanish. Following the steps of the above
proof, $x \to x'$ preserves proper time then \[ \eta_{\beta\delta} =
\eta_{\alpha\gamma} \PP{x'^\alpha}{x^\beta} \PP{x'^\gamma}{x^\delta}.\]
Taking a second derivative we get:
\begin{align*}
0 &= \eta_{\alpha\gamma}\FR{\di^2 x'^\alpha}{\di x^\veps\di x^\beta}
\PP{x'^\gam}{x^\delta} +\eta_{\alpha\gamma}\PP{x'^\alpha}{x^\beta}
\FR{\di^2 x'^\gam}{\di x^\veps\di x^\delta}\\
&= \eta_{\alpha\gam}x'^\alpha_{\ \ ;\veps\beta} x'^\gam_{\ \ ;\delta}
+\eta_{\alpha\gam}x'^\alpha_{\ \ ;\beta} x'^\gam_{\ \ ;\veps\delta}
\end{align*}
Now we do a \textbf{magical} transformation. We add this equation with
$\veps$ and $\beta$ swapped, and subtract this equation but with $\veps$
and $\delta$ swapped. This will give:
\[0=2\eta_{\alpha\gam} x'^\alpha_{\ \ ;\veps\beta}x'^\gam_{\ \ ;\delta}\]
Since $\eta$ is invertible and so is $x'^\gam_{\ \ ;\delta}$ by assumption,
it follows that $x'^\alpha_{\ \ ;\veps\beta} = 0$ for any
$\alpha,\beta,\veps$. 
\end{remark}

\begin{remark} In the last remark, we took $d\tau^2$ invariant. If we only
require that the equation $d\tau = 0$ (that is, the light-cone) to be
invariant then we would have obtained the conformal group. \colr{Check
this and desribe this and the last remark geometrically.}
\end{remark}

Here are some of the commonly used names associated with different types of
Lorentz transformations.
\begin{itemize}
    \item \textbf{Inhomogenous Lorentz Group/Poincar\'e Group} $\{x'^\alpha =
    \Lambda^\alpha_\beta x^\beta + a^\alpha\ |\ \Lambda\in\ms L\}$
    \item \textbf{Homogenous Lorentz Group:} $\{x'^\alpha =
    \Lambda^\alpha_\beta x^\beta\ |\ \Lambda\in\ms L\}$
    \item \textbf{Proper} (in)homogenous Lorentz groups: $\Lambda^0_0
        \ge1;\ \ \mathrm{det} \Lambda = 1$.
\end{itemize}
Using $\eta_{\alpha\beta}\Lambda^\alpha_\gam\Lambda^\beta_\delta =
\eta_{\gam\delta}$ with $\gam=\delta=0$ we have:
\[(\Lambda^0_0)^2 = 1+ \sum_{i=1}^3 (\Lambda^i_0)^2 \ge 1\ \text{and}\ 
(\mathrm{det}\ \Lambda)^2 = 1\]
This means that the \emph{proper} Lorentz groups are the connected
components of the identity $\Lambda^\alpha_\beta = \delta^\alpha_\beta$.
There are Lorentz transformations that involve \emph{space inversion}:
$\mathrm{det}\ \Lambda = -1, \Lambda^0_0 \ge1$; and that involve \emph{time
reversal}: $\mathrm{det}\ \Lambda = -1, \Lambda^0_0 \le 1$. \colr{What does
Weinberg mean when he says that space inversion is not an exact symmetry of
nature and why is time reversal suspected not to be one as well?}

\subsection{Boosts}
One may wonder what are the differences between the Galilean group and the
Poincar\'e group. The former can be recovered by taking $\Lambda^0_0 = 1$
and $\Lambda^0_i = \Lambda^i_0 = 0$. The Poincar\'e enjoy the extra feature
of \textbf{boosts}. Suppose there are two observers $O$ and $O'$ watching a
particle move. $O$ sees it at rest, while $O'$ sees it moving away at $\bm
v$. Then, since $d\bm x = 0$:
\begin{align*}
    dx'^i &= \Lambda^i_0 dt\\
    dt'   &= \Lambda^0_0 dt
\end{align*}
Dividing $dx'^i/dt'$ gives $\bm v'$:
\begin{align}
\Lambda^i_0 = \bm v\Lambda^0_0.
    \label{firstgammarelation}
\end{align}
In the defining relation for $\Lambda$,
$\eta_{\alpha\beta}\Lambda^\alpha_\gam\Lambda^\beta_\delta =
\eta_{\gam\delta}$, we can set $\gam=\delta=0$ and obtain:
\begin{align}
-1 = -(\Lambda^0_0)^2+\sum_{i=1}^3 (\Lambda^i_0)^2.
\label{secondgammarelation}
\end{align}
Plugging in the first relation, \eqref{firstgammarelation}, into equation
\eqref{secondgammarelation}, solving for $\Lambda^\mu_0$ describing the
boost:
\begin{align}
    \Lambda^0_0 = \gam\ \ \text{and}\ \ \Lambda^i_0 = \gam v_i.
    \label{boost}
\end{align}
A convenient choice, although not the only one, for $\Lambda^i_{\ \ j}$ is
the following:
\begin{align}
    \Lambda^i_{\ \ j}&=\delta_{ij}+v_iv_j\FR{\gam-1}{\bm v^2}\\
    \Lambda^0_{\ \ j}&= \gam v_j
    \label{definingLambda}
\end{align}

\section{Time Dilation}

\end{document}


