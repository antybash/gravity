%
\documentclass[11pt]{report}

\usepackage{graphicx}
\usepackage{braket}
\usepackage{mdframed}
\usepackage{bm}
\usepackage{amssymb}
\usepackage{amsthm}
\usepackage{amsmath}
\usepackage{enumerate}
\usepackage{amsmath}
\usepackage{mathrsfs}
\usepackage{multicol}
\usepackage{verbatim}
\usepackage{dsfont}
\usepackage{color}

%\usepackage[T1]{fontenc}
%\usepackage[euler-digits,small]{eulervm}
%\usepackage[sc,osf]{mathpazo}
%\linespread{1.025}
%\usepackage[utf8]{inputenc}

\usepackage{geometry}
\geometry{
letterpaper,
lmargin=2cm,
rmargin=2cm,
tmargin=2cm,
bmargin=2cm,
footskip=12pt,
headheight=12pt}

\newcommand{\iden}{\mathds{1}}

\newcommand{\Z}{\mathbb Z}
\newcommand{\T}{\mathbb{T}}
\newcommand{\lcm}{\text{lcm}}
\newcommand{\NN}{\mathbb{N}}
\newcommand{\QQ}{\mathbb{Q}}
\newcommand{\RR}{\mathbb{R}}
\newcommand{\CC}{\mathbb{C}}
\newcommand{\KK}{\mathbb{K}}
\newcommand{\DD}{\mathbb{D}}
\newcommand{\VV}{\mc {V}}
\newcommand{\WW}{\mc {W}}
\newcommand{\FF}{{\mathbb{F}}}
\newcommand{\tat}{\text}
\newcommand{\ti}{\to\infty}
\newcommand{\mc}{\mathcal}
\newcommand{\ms}{\mathscr}
\newcommand{\mf}{\mathfrak}
\newcommand{\mbb}{\mathbb}
\newcommand{\cnj}{\overline}
\newcommand{\veps}{\varepsilon}
\newcommand{\sg}{\sigma}

\newcommand{\restr}{\upharpoonright}
\newcommand{\FR}[2]{\frac{#1}{#2}}
\newcommand{\PFR}[2]{\left(\frac{#1}{#2}\right)}
\newcommand{\SFR}[2]{\sqrt{\frac{#1}{#2}}}

\newcommand{\ten}{\otimes}

\newcommand{\Align}[1]{\begin{align*}#1\end{align*}}

\theoremstyle{plain}
\newtheorem{thm}{Theorem}[section] 
\newtheorem{lem}[thm]{Lemma}

\theoremstyle{definition}
\newtheorem{defn}{Definition}
\newtheorem{proposition}{Proposition}
\newtheorem{conj}{Conjecture}
\newtheorem{ex}{Example}

\theoremstyle{remark}
\newtheorem*{remark}{Remark}
\newtheorem*{note}{Note}
\newtheorem{case}{Case}
\newtheorem*{claim}{Claim}

\DeclareMathOperator{\spec}{Spec}
\DeclareMathOperator{\Card}{card}
\DeclareMathOperator{\Span}{span}
\DeclareMathOperator{\rank}{rank}
\DeclareMathOperator{\real}{Re}
\DeclareMathOperator{\diam}{diam}
\DeclareMathOperator{\id}{id}
\DeclareMathOperator{\GL}{GL}

\def\sm{\setminus}
\def\seq{\subseteq}
\def\ii{\item}
\def\bE{\begin{enumerate}}
\def\eE{\end{enumerate}}
\def\bP{\begin{pmatrix}}
\def\eP{\end{pmatrix}}

\renewcommand{\empty}{\varnothing}
\newcommand{\hv}[1]{\hat{\textbf{#1}}}
\newcommand{\oti}[2]{#1_{#2=1}^\infty}
\newcommand{\lam}{\lambda}
\newcommand{\om}{\omega}
\newcommand{\Om}{\Omega}
\newcommand{\gam}{\gamma}
\newcommand{\di}{\partial}

\newcommand{\ha}{\hat a}
\newcommand{\had}{\hat a^\dagger}

\newcommand{\colr}[1]{ {\color{red}  #1 } }
\newcommand{\colb}[1]{ {\color{blue} #1 } }
\newcommand{\colnb}[1]{ {\color{NavyBlue} #1 } }
\newcommand{\colm}[1]{ {\color{Fuchsia} #1 } }

\date{\today}
\title{Gravitation and Cosmology Notes}
\author{}

\begin{document}
\maketitle

\chapter{Special Relativity}

The axioms of Special Relativity are:
\begin{enumerate}
    \item The laws are invariant under change of inertial reference frames.
    \item In inertial reference frames, there is an absolute speed of
        signal propogation $c=1$.
\end{enumerate}

\section{Lorentz Transformations}
Let $x^\mu,x'^\mu$ be two coordinate systems and $c=1$. For now we work in the
classical vacuum and here experiments\footnote{Is there a better explanation?}
point us to the fact that light travels at the absolute speed $c$.
Suppose $A,B$ are spacetime events representing emmission and absorption of
light. By the second postulate, the distance between these two points is the
same in both reference frames, and in particular, the quantity 
$(\Delta x^0)^2-(\Delta x^1)^2-(\Delta x^2)^2-(\Delta x^3)^2 = 0 = (\Delta
x'^0)^2-(\Delta x'^1)^2-(\Delta x'^2)^2-(\Delta x'^3)^2$.
This leads us to the definition of the \textbf{proper time}:
\[d\tau^2 := dt^2 - d\bm x^2 = -\eta_{\alpha\beta}dx^\alpha dx^\beta.\]

The notation $dx^\alpha dx^\beta$, in the mathematical sense, is a simple tensor
$dx^\alpha\ten dx^\beta$ and \emph{not} the symmetric tensor. Using the Einstein
summation convention and the fact that our metrics are always \emph{symmetric},
this means that the contraction $\eta_{\alpha\beta}dx^\alpha dx^\beta
=\eta_{00}dx^0\ten dx^0+\eta_{10}dx^1\ten dx^0+\eta_{01}dx^0\ten{dx}^1+\cdots
=\eta_{00}dx^0\ten dx^0+\eta_{10}(dx^1\ten dx^0+dx^0\ten{dx}^1)+\cdots$. Now we
return back to proper time and see how far we can run with the concept.

\begin{proposition} Let $\ms L \seq \GL_\RR(3,1)$ denote the subgroup of linear
operators that fix proper time: $d\tau'^2 = d\tau^2$. Then the set $\ms L$ can
be described concretely:
\[ \ms L = \{ \Lambda\ |\ \Lambda^\alpha_\gamma
\Lambda^\beta_\delta\eta_{\alpha\beta} = \eta_{\beta\delta}\}.\]
\end{proposition}


\end{document}


